\chapter{EMG-Signale und Merkmalsextraktion}
\label{chap:emg}

% Elektromyographie ist ein Verfahren zur Messung elektrischer Muskelaktivität ✅
% Impuls wird vom Hirn bis zur motorischen Endplatte weitergeleitet, auf der unterschiedliche Ionen ausgeschüttet werden ✅
% Es entstehen kleine messbare Zuckungen im Muskel ✅
% Messung entweder intramuskulär (aufwändig, aber weniger Rauschen) oder von außen mit Trocken- oder Gelelektrode ✅
% In der Praxis vor allem äußere Messung mit Trockenelektroden relevant ✅

Nach \cite{bischoff2015emg} wird elektrische Muskelaktivität wird durch kleine Zuckungen im Muskel ausgelöst, sobald ein Signal vom Gehirn aus die Anweisung für eine Bewegung sendet. Die Messung dieser Signale bezeichnet man als Elektromyographie. Hierbei ist die Art der Messung essenziell für eine qualitativ hochwertige Auswertung der Signale. So unterscheidet man beispielsweise zwischen intramuskulärer Messung und der Messung über extern an der Haut angebrachte Elektroden. Während die Messung von EMG-Signalen über intramuskuläre Elektroden genauer ist als eine duch die Anbringung der Elektroden auf der Haut, da externe Störfaktoren wie die Beschaffenheit der Haut ausgeschlossen werden können, sind diese alltagstauglicher und finden somit in der Praxis häufiger Anwendung (\cite{bischoff2015emg}). Hier kann man laut \cite{Allard2019} weiterhin zwischen Gel- und Trockenelektroden unterscheiden. Bei diesen verhält es sich ähnlich wie bei dem Vergleich der intramuskulären und äußeren Messung. Während Gelelektroden rauschfreiere Signale liefern, sind diese aufgrund der vorbereitenden Maßnahmen für eine alltägliche Nutzung in der Regel nicht geeignet (\cite{Allard2019}). Deshalb tendiert man häufig auch hier in der Anwendung zu den alltagstauglichen Trockenelektroden. Die Messung von Elektromyographischen Signalen über an der Haut angebrachte Elektroden bezeichnet man dabei als \textit{Oberflächen-Elektromyographie} (sEMG) (\cite{bischoff2015emg}).

Um nun mit Hilfe von Algorithmen oder Neuralen Netzen die sEMG-Signale zu klassifizieren, müssen Merkmale extrahiert werden. Der Zweck von Merkmalen ist es nach \cite{zecca2002control}, vergleichbare Eigenschaften der Signale zu erhalten und gleichzeitig für die Klassifikation überflüssige Daten herauszufiltern. Das ist relevant, da für eine erfolgreiche Klassifikation sowohl ausreichend aussagefähige Daten als auch eine hohe rechnerische Effizienz von Nöten sind.  Meistens muss diese Merkmalsextraktion vor dem Einspeisen der aufgezeichneten Daten ins Netz geschehen (\cite{zecca2002control}). Eine Ausnahme hiervon bilden die sogenannten Convolutional Neural Networks, die die Merkmalsextraktion innerhalb der Netzwerkarchitektur vornehmen (\cite{Allard2019}). Sämtliche in dieser Arbeit besprochenen Klassifikatoren benötigen bereits extrahierte Merkmale als Eingabewerte, weshalb im Folgenden auf die verschiedenen Bereiche und Unterschiede in der Merkmalsextraktion eingegangen wird.

So unterscheidet man bei der Merkmalsextraktion zwischen der Extraktion aus dem Zeit-, Frequenzbereich oder einer Mischung aus beiden (\cite{zecca2002control}).

Laut \cite{zecca2002control} finden die Merkmale aus dem Zeitbereich (TD) aufgrund ihrer hohen rechnerischen Effizient häufig Anwendung. Dieser stellt durch einen Graphen die Entwicklung von elektrischer Spannung in Abhängigkeit von Zeit dar. Die rechnerische Effizienz ist darauf zurückzuführen, dass es für die Merkmalsextraktion aus dem TD keiner Transformation bedarf (\cite{zecca2002control}). Für die Klassifikation von sEMG Signalen werden in dieser Arbeit die Merkmale des \textit{mean absolute value}, des \textit{zero crossings}, des \textit{slope sign changes} und der \textit{waveform length} entnommen, da diese in der Forschung bereits zu guten Ergebnissen führten (\cite{Engelhart2003}, \cite{Kaufmann2013}). Die extrahierten Merkmale werden im Detail im nächsten, den Datensatz thematisierenden, Kapitel beschrieben.

Der \textit{Frequenzbereich} (FD) betrachtet nach \cite{zecca2002control} anders als der TD die elektrische Spannung in Abhängigkeit der Frequenz. Um aus dem FD Merkmale zu extrahieren, müssen die Daten aus diesem zunächst transformiert werden. Das macht die Merkmalsextraktion rechnerisch intensiver als die Extraktion aus dem TD, weshalb diese Methode seltener für die Klassifikation von sEMG-Signalen in ES verwendet wird (\cite{zecca2002control}).

Eine weitere Möglichkeit der Merkmalsextraktion ist die Kombination von Zeit- und Frequenzbereich (\cite{zecca2002control}).
% Nochmal in den Quellen überprüfen
Hier wird die elektrische Spannung nicht nur in Abhängigkeit von Zeit, sondern auch in Abhängigkeit der Frequenz betrachtet. Daraus resultiert eine größere Vielfalt an Merkmalen, allerdings ist für die Extraktion trotzdem eine Transformation nötig, weshalb dieser Prozess rechnerisch ebenfalls intensiver ist als die direkte Extraktion aus dem Zeitbereich (\cite{zecca2002control}). 