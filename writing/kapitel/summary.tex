\chapter{Fazit und Ausblick}
\label{chap:fazit}

Kern der Forschungsarbeit war der Vergleich zwischen Rekurrenten Neuralen Netzen und state-of-the-art Klassifikatoren, um einen Referenzwert für weitere Forschung in diesem Bereich zu schaffen. Dabei wurden die Besonderheiten und Problemstellungen, die sich durch die Arbeit mit RNN-Architekturen ergeben, aufgezeigt. Rekurrente Neurale Netze wurden dabei erstmals auf den \glqq Forearm 121 EMG\grqq{} Datensatz angewandt. Hierfür wurde eigens für diese Arbeit eine Implementierung der Architekturen vorgenommen und die Netzwerke anhand des Datensatzes trainiert und ausgewertet. Weiterhin wurden state-of-the-art Klassifikatoren implementiert und mit den untersuchten RNN-Architekturen verglichen. Dabei zeigte sich, dass RNN-Architekturen valide Klassifikatoren für EMG-Daten sind und im Vergleich zu gewöhnlichen Klassifikatoren eine signifikant höhere Klassifikationsgenauigkeit aufweisen. Zwar brauchen ältere RNN-Architekturen wie LSTM das achtfache der Zeit, um eine Steigerung der Klassifikationsgenauigkeit um zehn Prozentpunkte zu gegenüber gewöhnlicher Klassifikatoren zu erzielen, allerdings ermöglichen junge Architekturen wie das GRU gleiche Präzision in einem Drittel der Zeit des LSTM. So lässt sich als Fazit ableiten, dass insbesonders moderne RNN-Architekturen wie die GRU mit einiger Optimierung das Potenzial haben könnten, eine präzise Klassifikation von EMG-Signalen in einer ähnlichen Zeit, jedoch mit einer besseren Klassifikationsgenauigkeit, durchzuführen, als selbst präzise gewöhnliche Klassifikatoren wie die SVM.

Das anfängliche Ziel der Forschungsarbeit und die Frage, wie RNN im Vergleich zu gewöhnlichen state-of-the-art Klassifikatoren klassifizieren, lassen sich nun beantworten. Während klassische RNN-Architekturen teilweise signifikant länger für die Klassifikation brauchten als gewöhnliche state-of-the-art Klassifikatoren, schließt sich nun die Lücke in der Trainingsdauer durch moderne Klassifikatoren wie die GRU langsam.

Das macht moderne RNN-Architekturen zu einem vielversprechenden Forschungsansatz, der in Zukunft in Verbindung mit der Klassifikation von EMG-Signalen weiter erforscht werden sollte. Hier sollte der Schwerpunkt auf modernen Architekturen wie dem GRU liegen, da sich im Training von gewöhnlichen RNN auf Grund von Vanishing und Exploding Gradients häufig Probleme offenbaren, die sich durch Verwendung moderner RNN-Architekturen einfach vermeiden lassen. Gleichzeitig dürfen gewöhnliche Klassifikatoren wie die SVM nicht vernachlässigt werden, da diese in vergleichsweise kurzer Zeit und mit wenigen Hyperparametern häufig bereits vergleichbare Ergebnisse erzielen können. Auch einfache Klassifikatoren wie die Entscheidungsbäume könnten aufgrund ihrer leichten Interpretierbarkeit und guter Klassifikationsgenauigkeit ein interessanter Ansatz für weitere Forschung sein. Sie könnten unter anderem zum Verständnis der Klassifikation von EMG-Signalen beitragen.

Für praktische Anwendungsfälle wie die Prothesentechnik zeigt die Klassifikation durch RNN-Architekturen ein großes Potenzial auf. Zwar ist eine kurze Trainingsdauer für eine praktische Anwendung wichtig, eine hohe Klassifikationsgenauigkeit allerdings essenziell. Somit bieten sich hier die untersuchten RNN-Architekturen gut für weitere Forschung in der Zusammenarbeit mit Anwendern an.

Gleichzeitig ist anzumerken, dass in vorgehenden Untersuchungen teilweise bessere Ergebnisse in der Genauigkeit einiger untersuchter Klassifikatoren wie den SVM oder den gewöhnlichen RNN, erzielt wurden. Diese Ergebnisse könnten weitere Potenziale offenbaren und sollten für zukünftige Forschung miteinbezogen werden.

Konkludierend ist aufbauend auf den Ergebnissen dieser Arbeit zu sagen, dass die Präzision und Geschwindigkeit moderner Klassifikatoren durch weitere Forschung vermutlich bald in der Lage sein wird, die alltäglichen Einschränkungen auf Prothesen angewiesener Menschen nachhaltig zu lösen. 
