% Die Titelseite der Arbeit

\begin{titlepage}

\begin{center} % zentrieren

  % Logo der Universität Mainz
  \begin{figure}[ht]
    \centering
    \includegraphics{grafiken/jogu_logo.eps}
  \end{figure}

  % Vertikaler Zwischenraum
  \bigskip
  \vfill
  \begin{framed}
  % Titel der Arbeit und Typ der Arbeit, umrandet
    \begin{center}
      \textsc{{\Large Klassifikation elektrischer Muskelaktivität durch Tiefe Neuronale Netze\\}}
                                % Letztes \\ ist wichtig, beginnt eine neue Zeile f{\"u}r die Art der Arbeit

      \bigskip

                                % Art der Arbeit, ggf. auszutauschen gegen Seminar- oder Doktorarbeit
      \textbf{Bachelorarbeit}
    \end{center}
    \end{framed}
    \vfill
    \vfill

  % Daten des Erstellers, Einreichungsdatum
  % in einer Tabelle ausgerichtet
  \begin{tabular*}{0.62\textwidth}{r@{\extracolsep{\fill}}l}
    eingereicht im: & August 2020\\\\
    von: & Alexander Seidmann\\
    & geboren am 22. Januar 1998\\
    & in Frankfurt am Main\\
    \\
    Matrikelnummer: & 2740392\\
    \\
    Betreuer: & Dr. Paul Kaufmann
  \end{tabular*}
  \vfill
  \vfill

  % Unten: Kontaktdaten des Lehrstuhls f�r Wirtschaftsinformatik und BWL

  \rule{\textwidth}{.4pt}\\ % vertikale Linie
  Johannes Gutenberg-Universität Mainz\\
  Fachbereich Rechts- und Wirtschaftswissenschaften\\
  Lehrstuhl für Wirtschaftsinformatik und BWL\\
  Telefon: +49 6131 39-22734, Fax +49 6131 39-22185\\
  Internet: \url{http:///wi.bwl.uni-mainz.de}
\end{center}

\end{titlepage} % Ende des Titelblatts

